\section{RESULTS} \label{results}

We first report here the results obtained on the two 2D datasets. In fig.\ref{fig:2Dplots} points generated after 3000 epochs are displayed, while in fig.\ref{fig:2Dlosses} the losses for both D and G are plotted, as functions of the epochs, i.e. cycles over the entire dataset.
As we can see in fig.\ref{fig:2Dlosses}, the losses converge to the theoretical optimum loss values, which are $-\log(\alpha)$ for G and $-\alpha\cdot\log(\alpha) - (1-\alpha)\cdot\log(1-\alpha)$ for D.
This proves that the results obtained in our simulations are consistent with our theoretical analysis. Furthermore, from inspection of fig.\ref{fig:2Dplots} we select $\alpha=\{0.3,0.5,0.7\}$ as the best values. Among these, a faster convergence to the equilibrium point is obtained with $\alpha=0.3$, even though the difference from $\alpha=0.5$ isn't that relevant.

When training the model on the MNIST dataset, results for the losses, as shown in fig.\ref{fig:MNISTlosses}, are consistent with the losses on the 2D datasets, not for convergence values but for their relations. Generated images are displayed in fig.\ref{fig:MNISTplots} at the end of epochs $\{1,20,40,60,80,100\}$. What can be seen here is that for each $\alpha$ satisfactory results can be achieved, even though with different speeds (in terms of epochs). From our analysis on the 2D datasets, we expect that also in this case the optimal value of $\alpha$ is in the range $[0.3,0.5]$.

\begin{figure*}
	\includegraphics*[width=\textwidth]{./plots/pdf_2D_fit.png}
	\caption{blue points are samples from the true distribution $p_{data}$, red dots are 100 samples generated from $p_g$, after the model was trained on 3000 epochs with mini-batch size 128 and a noise dimensionality of 10.}
	\label{fig:2Dplots}
\end{figure*}
\begin{figure*}
	\includegraphics*[width=\textwidth]{./plots/losses_2D.eps}
	\caption{losses for 2D datasets of both G and D for all values of $\alpha$. Dashed lines in 'G Loss' plots represent the theoretical optimal values $-\log(\alpha)$}
	\label{fig:2Dlosses}
\end{figure*}
\begin{figure*}
	\includegraphics*[width=\textwidth]{./plots/losses_MNIST.eps}
	\caption{losses for 2D datasets of both G and D for all values of $\alpha$. Dashed lines in 'G Loss' plots represent the theoretical optimal values $-\log(\alpha)$}
	\label{fig:MNISTlosses}
\end{figure*}
\begin{figure*}
	\includegraphics*[width=\textwidth]{./plots/MNIST_digits_evoling.eps}
	\caption{blue points are samples from the true distribution $p_{data}$, red dots are 100 samples generated from $p_g$, after the model was trained on 3000 epochs with mini-batch size 128 and a noise dimensionality of 10.}
	\label{fig:MNISTplots}
\end{figure*}