\documentclass[article,10pt]{IEEEtran}
%\documentclass[conference,draft,onecolumn]{IEEEtran}
% useful packages, copy and paste from diff sources

\usepackage[english]{babel}
\usepackage[T1]{fontenc}
%\usepackage{cite,url,color} % Citation numbers being automatically sorted and properly "compressed/ranged".
\usepackage{graphics,amsfonts}
\usepackage{subfigure}
\usepackage{epstopdf}
\usepackage[pdftex]{graphicx}
\usepackage[cmex10]{amsmath}
\DeclareMathOperator*{\argmax}{arg\,max}
\usepackage{algorithm}
\usepackage{algorithmicx}
\usepackage{algpseudocode}
\usepackage{amsmath}
\usepackage{amssymb}
\usepackage{hyperref}
%\usepackage[ansinew]{inputenc} 
\usepackage{xcolor}
\usepackage{mathtools}
\usepackage{graphicx}
\usepackage{caption}
%\usepackage{subcaption}
\usepackage{import}
\usepackage{multirow}
\usepackage{cite}
%\usepackage[export]{adjustbox}
\usepackage{breqn}
\usepackage{mathrsfs}
\usepackage{acronym}
%\usepackage[keeplastbox]{flushend}
\usepackage{setspace}
\usepackage{stackengine}
\usepackage{tikz}
\usepackage{natbib}

\usetikzlibrary{calc, positioning}
\tikzset{
	every node/.style={
		rectangle,
		draw,
		minimum width=3.5cm,
		minimum height=1cm
	},
	>=latex,
}

% Also, note that the amsmath package sets \interdisplaylinepenalty to 10000
% thus preventing page breaks from occurring within multiline equations. Use:
\interdisplaylinepenalty=2500
% after loading amsmath to restore such page breaks as IEEEtran.cls normally does.
\usepackage[utf8]{inputenc}
% Useful for displaying quotations
%\usepackage{csquotes}
% Compact lists
%\let\labelindent\relax
%\usepackage{enumitem}
\usepackage{fancyhdr}
\fancyfoot[C]{\thepage}
%tikz figures
%\usepackage{tikz}
%\usetikzlibrary{automata,positioning,chains,shapes,arrows}
%\usepackage{pgfplots}
%\usetikzlibrary{plotmarks}
%\newlength\fheight
%\newlength\fwidth
%\pgfplotsset{compat=newest}
%\pgfplotsset{plot coordinates/math parser=false}

\usepackage{array}
% http://www.ctan.org/tex-archive/macros/latex/required/tools/
%\usepackage{mdwmath}
%\usepackage{mdwtab}
%mdwtab.sty	-- A complete ground-up rewrite of LaTeX's `tabular' and  `array' environments.  Has lots of advantages over
%		   the standard version, and over the version in `array.sty'.
% *** SUBFIGURE PACKAGES ***
%\usepackage[tight,footnotesize]{subfigure}

\usepackage{subfigure}
\usepackage[top=1.5cm, bottom=2cm, right=1.6cm,left=1.6cm]{geometry}
\usepackage{indentfirst}

\usepackage{times}
% make sections titles smaller to save space
%\usepackage{sectsty}
%\sectionfont{\large}
% enable the use of 'compactitem', a smaller 'itemize'
%\usepackage{paralist}

% MP
% to split equations using dmath env
\usepackage{breqn}
% nice rules in tables
\usepackage{booktabs}


%\setlength\parindent{0pt}
\linespread{1}

% MC
\newcommand{\MC}[1]{\textit{\color{red}MC says: #1}}
\newcommand{\AZ}[1]{\textit{\color{blue}AZ says: #1}}
\newcommand{\MP}[1]{\textit{\color{green}MP says: #1}}

\usepackage{placeins}


\setcounter{page}{1}
\usepackage{multirow}
%\mainmatter
%%%%%%%%%%%%%%%%%%%%%%%%%%%%%%%%%%%%%%%%%%
\begin{document}
	%%%%%%%%%%%%%%%%%%%%%%%%%%%%%%%%%%%%%%%%%%
	\title{A deep introspection on Generative Adversarial Networks}
	
	\author{\IEEEauthorblockN{Riccardo Lincetto, Guglielmo Camporese}
		
		\IEEEauthorblockA{Department of Information Engineering, University of Padova -- Via Gradenigo, 6/b, 35131 Padova, Italy \\
			emails: riccardo.lincetto, guglielmo.camporese @studenti.unipd.it
	}}
	
	\maketitle		
	
	\begin{abstract}
		GANs, namely Generative Adversarial Networks, are a hot topic nowadays. These models have the ability of generating good quality data, learning their distributions on a training set. Image generation in particular benefited from this framework, thanks also to the ease of assessment of the results obtained. Here we review the analysis of GAN models, recalling the necessary results from game theory, and propose a parametrization of the problem, which is then assessed analytically and by means of simulation, on image generation with MNIST dataset. With this parametrization it seems that in some cases it is possible to speed up the training phase.
	\end{abstract}
	
	\section{INTRODUCTION} \label{intro}

Since its rise, deep learning had a great impact on discriminative models. Generative models instead were not affected by this innovation at first, but this trend changed with the introduction of Generative Adversarial Networks (GAN), a powerful framework first introduced in \cite{NIPS2014_5423}. Since then, GAN gained more and more momentum because of the ability of training \textit{deep generative models}, avoiding some of the difficulties encountered in other frameworks \cite{DBLP:journals/corr/Goodfellow17}.

GAN is a sub-class of generative models that use, generally two, \textit{neural networks}: given a training set of sample data, distributed according to a probability density function (pdf) $p_{data}$, the purpose of GAN is to generate samples according to a distribution $p_g$, that mimics $p_{data}$, without explicitly defining it.
As suggested by the name, this is achieved by putting in competition two entities: a generator (G) and a discriminator (D). The task of G is to generate data that can be regarded as true by D, while D has the purpose of correctly distinguishing real from fake data. The classical real-life analogy with this process involves counterfeiters trying to produce fake currency and the police trying to detect it. A graphical representation of the process is depicted in fig.\ref{fig:game}.
This kind of interaction between the two entities can naturally be modeled with a game theoretical approach, where each player has its own strategies and payoffs, but in this paper we will rather talk about costs, as will be discussed in \ref{relatedwork}.
This framework anyway has a major drawback: computing a Nash Equilibrium (NE) requires assuming that the networks have enough capacity, i.e. it has to be done in the non-parametric limit for the networks \cite{NIPS2014_5423}. This implies that the equilibrium point found in practice can be different from the theoretical one, because of intrinsic limits imposed by implementing a network with a finite set of parameters.

In this paper we review some of the literature and explain our need to go back to the origins of GAN, implementing our own version of the code and simulating different scenarios, where discriminator is passed different fake-to-true ratios of images.

The remainder of the paper is organized as follows: a brief overview of the literature is presented in \ref{relatedwork}; a description of our work is then presented in \ref{expsetting}; the obtained results are presented in \ref{results}; finally we discuss our conlcusions in \ref{conclusions}.
	\section{RELATED WORK} \label{relatedwork}

Since their appearance in literature, GANs have been successfully applied to problems of image generation, editing and semi-supervised learning \cite{DBLP:journals/corr/RadfordMC15} \cite{DBLP:journals/corr/ZhangXLZHWM16}. \textit{The results obtained with this technique were so good that they captured the attention of a great number of researchers, leading to a proliferation of various flavors of GAN, each performing better than the others on a specific domain.} It's difficult anyway to understand how to compare different GAN models, because of the lack of a consistent metric and the different architectures with which networks can be designed, which for each project are related to the corresponding computational budget. A tentative to define some guidelines to avoid these problems, together with a fair and comprehensive comparison of state-of-the-art GANs, is discussed in \cite{46506}[MISSING ACCENTS ON REF]: what emerges here is that the computational budget plays a major role, allowing bad algorithms to outperform good ones if given enough time; plus, despite many claims of algorithms being superior to the original ones, there is no empirical evidence that they are across all datasets, in fact the original model outperforms them in most of them. [INSERT TABLE WITH LOSSES].
We thus report here a formalization of the original problem \cite{NIPS2014_5423}, modeling it with a game theory approach in a more precise way, as done in \cite{2017arXiv171200679O}.

Deep Convolutional GAN (DCGAN) can be naturally represented as a 2-player zero-sum infinite game: the player entities, G and D, are two neural networks, each with a given architecture defining its depth and width; pure strategies are the different combinations of weights for the given architecture, and it is immediate to notice that these numbers are infinite; payoffs are defined as the opposite of their loss functions, which are also used for the training phase of the networks.
An infinite game doesn't guarantee the existence of a NE, not even in mixed strategies, nor it allows to find it as the value of the game, i.e. maximinimizing the payoffs. As pointed out in  \cite{2017arXiv171200679O} though, when using floating point numbers, the number of combinations of strategies becomes finite, albeit very large: there is then a NE, at least in mixed strategies, which however is hard to compute. It has to be kept into account also the fact that the optimization of a neural network can lead to a Local NE (LNE) because the problem is non-convex.

In strategic form, GAN games can be described by a tuple $(\{G, D\}, \{S_G, S_D\}, \{u_G, u_D\})$, where $S_G$ and $S_D$ are the (finite) sets of strategies while $u_G$ and $u_D$ are the corresponding payoffs, for G and D respectively.
The loss functions used for DCGAN training are: [CORREGGERE]
\begin{align*}
	L_D = E_{x \sim p_d}[\log(D(x))]+E_{x \sim p_g}[\log(1-D(x))]\\
	L_G = E_{x \sim p_d}[\log(D(x))]+E_{x \sim p_g}[\log(1-D(x))]
\end{align*}
which correspond to the binary cross-entropy loss functions. Each of the two players should then minimize its own loss function, which is equivalent to maximizing its payoff. The problem is then reduced to a maximin problem, where the value of the game is defined as 
	\section{EXPERIMENTAL SETTING} \label{expsetting}

For our project, we implemented a version of NSGAN which can be found at \cite{bibid}. Usually the training of neural networks occurs after mini-batches of data are passed to them: for the discriminator network D, in a mini-batch there are the same number of true and of fake images. We instead carried out the training with different configurations, where D is passed different true-to-fake data ratios: this is done in practice defining a parameter $\alpha$ that represents the portion of true data with respect to the size of the mini-batch. This results in a parametrisation of the loss function as follows: [FOR D]
\begin{align*}
	\alpha\cdot\mathbb{E}_{x \sim p_{data}(x)}[\log D(x)] + (1-\alpha)\cdot\mathbb{E}_{z \sim p_{z}(z)}[\log (1-D(G(z)))]
\end{align*}
Performing the same analysis as before, we found that the optimal discriminator should have the form:
\begin{align*}
D_{\alpha}^*(x) = \frac{\alpha \cdot p_{data}(x)}{\alpha \cdot p_{data}(x) + (1-\alpha)\cdot p_g(x)}
\end{align*}
and that the loss of the generator is minimized, as before, when
\begin{align*}
p_g = p_{data},
\end{align*}
from which the optimal D can be rewritten as
\begin{align*}
D_{\alpha}^*(x) = \alpha.
\end{align*}
To support these results, it can be noticed also that when setting $\alpha = 0.5$, i.e. in the same case previously analysed with the same amount of true and fake images in a mini-batch, the results are consistent.

Before testing this model on MNIST dataset, we tried it on different ad-hoc 2D distributions, with the following parameters:


The results obtained with them are reported in \ref{results}, together with the same plot obtained for the MNIST dataset.
	\section{RESULTS} \label{results}

We first report here the results obtained on the two 2D datasets. In fig.\ref{fig:2Dplots} points generated after 3000 epochs are displayed, while in fig.\ref{fig:2Dlosses} the losses for both D and G are plotted, as functions of the epochs, i.e. cycles over the entire dataset.
As we can see in fig.\ref{fig:2Dlosses}, the G loss converges to the theoretical optimum loss value, which is $-\log(\alpha)$, and the same happens for D, to the respective values.

It can be immediately noticed that for both data distributions, the highest loss for the discriminator is obtained with $\alpha=0.5$ which foretells the training of a good generator (notice that for this $\alpha$, as expected, $L^{(D)}$ converges to $\ln(2)=0.693$). Anyway also with $\alpha=\{0.3,0.7\}$ good results are achieved in this sense. From the generator's losses it can be noticed that, the lower $\alpha$, the faster the loss convergence to the final value. An insight on how well are points generated is given in FIG.X, where for different values of $\alpha$ we can see how points are generated after 3000 training epochs: these results confirm what expected, that is better performances from more balanced number of true images per mini-batch.

When applying the model to the MNIST dataset, similar results are obtained FIG.X, with losses for the discriminator approaching this time $0.5$ for $\alpha=\{0.3,0.5,0.7\}$. Also in this data space, as pointed out in the figure, when mini-batches are more balanced the D loss is higher, indicating better generators.
The evolution of how numbers are generated is depicted in FIG.X, where for each $\alpha$, every 20 and up to 100 epochs, a grid with 36 outputs are displayed. Also here for lower values of $\alpha$, the generator outputs acceptable images much faster, as observed for the 2D case.

\begin{figure*}
	\includegraphics*[width=\textwidth]{./plots/pdf_2D_fit.png}
	\caption{blue points are samples from the true distribution $p_{data}$, red dots are 100 samples generated from $p_g$, after the model was trained on 3000 epochs with mini-batch size 128 and a noise dimensionality of 10.}
	\label{fig:2Dplots}
\end{figure*}
\begin{figure*}
	\includegraphics*[width=\textwidth]{./plots/losses_2D.eps}
	\caption{losses for 2D datasets of both G and D for all values of $\alpha$. Dashed lines in 'G Loss' plots represent the theoretical optimal values $-\log(\alpha)$}
	\label{fig:2Dlosses}
\end{figure*}
\begin{figure*}
	\includegraphics*[width=\textwidth]{./plots/losses_MNIST.eps}
	\caption{losses for 2D datasets of both G and D for all values of $\alpha$. Dashed lines in 'G Loss' plots represent the theoretical optimal values $-\log(\alpha)$}
	\label{fig:MNISTlosses}
\end{figure*}
\begin{figure*}
	\includegraphics*[width=\textwidth]{./plots/MNIST_digits_evoling.eps}
	\caption{blue points are samples from the true distribution $p_{data}$, red dots are 100 samples generated from $p_g$, after the model was trained on 3000 epochs with mini-batch size 128 and a noise dimensionality of 10.}
	\label{fig:MNISTplots}
\end{figure*}
	\section{CONCLUSIONS} \label{conclusions}

In this report we presented a review of GAN framework, as presented in the original paper \cite{NIPS2014_5423}, with a game theoretical approach, going through the resolution method to find a Nash equilibrium. We then proposed a parametrization of the loss function by varying the percentage of true images used for the training in each mini-batch, adapting the analysis previously presented to this new case. Then we tried to apply our own implementation of the model to three different datasets, two of which are generated ad-hoc and used as toy examples, while the last one is the MNIST dataset, well known in literature for this kind of tasks.
Results obtained show that when mini-batches are more balanced, the generator model performs better, but also trainings with less true samples can be performed with good results, which opens the door to the use of GANs even for slightly smaller datasets.

Another possible way to organise mini-batches, that could be investigated in the future, is to introduce randomness in the percentage of true samples per mini-batch: $\alpha$ for example could stand for the probability of passing a true sample. The behaviour in this case should be similar, on average, to the one with $\alpha$ indicating a deterministic ratio.
	
	\bibliography{biblio}
	\bibliographystyle{plain}
	
	
\end{document}