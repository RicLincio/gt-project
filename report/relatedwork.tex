\section{RELATED WORK} \label{relatedwork}

Since their appearance in literature, GANs have been successfully applied to problems of image generation, editing and semi-supervised learning \cite{DBLP:journals/corr/RadfordMC15} \cite{DBLP:journals/corr/ZhangXLZHWM16}. \textit{The results obtained with this technique were so good that they captured the attention of a great number of researchers, leading to a proliferation of various flavors of GAN, each performing better than the others on a specific domain.} It's difficult anyway to understand how to compare different GAN models, because of the lack of a consistent metric and the different architectures with which networks can be designed, which for each project are related to the corresponding computational budget. A tentative to define some guidelines to avoid these problems, together with a fair and comprehensive comparison of state-of-the-art GANs, is discussed in \cite{46506}[MISSING ACCENTS ON REF]: what emerges here is that the computational budget plays a major role, allowing bad algorithms to outperform good ones if given enough time; plus, despite many claims of algorithms that are superior to the original ones, there is no empirical evidence that they are across all datasets, in fact the original model outperforms them in most datasets. [INSERT TABLE WITH LOSSES].

We thus present here a formalization of the original problem \cite{NIPS2014_5423}, modeling it with a game theory approach in a more precise way, as partially done in \cite{2017arXiv171200679O}.

Deep Convolutional GAN (DCGAN) can be naturally represented as a 2-player zero-sum game: