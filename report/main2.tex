\documentclass[article,10pt]{IEEEtran}
%\documentclass[conference,draft,onecolumn]{IEEEtran}
% useful packages, copy and paste from diff sources

\usepackage[english]{babel}
\usepackage[T1]{fontenc}
%\usepackage{cite,url,color} % Citation numbers being automatically sorted and properly "compressed/ranged".
\usepackage{graphics,amsfonts}
\usepackage{subfigure}
\usepackage{epstopdf}
\usepackage[pdftex]{graphicx}
\usepackage[cmex10]{amsmath}
\DeclareMathOperator*{\argmax}{arg\,max}
\usepackage{algorithm}
\usepackage{algorithmicx}
\usepackage{algpseudocode}
\usepackage{amsmath}
\usepackage{amssymb}
\usepackage{hyperref}
%\usepackage[ansinew]{inputenc} 
\usepackage{xcolor}
\usepackage{mathtools}
\usepackage{graphicx}
\usepackage{caption}
%\usepackage{subcaption}
\usepackage{import}
\usepackage{multirow}
\usepackage{cite}
%\usepackage[export]{adjustbox}
\usepackage{breqn}
\usepackage{mathrsfs}
\usepackage{acronym}
%\usepackage[keeplastbox]{flushend}
\usepackage{setspace}
\usepackage{stackengine}
\usepackage{tikz}
\usepackage{natbib}

\usetikzlibrary{calc, positioning}
\tikzset{
	every node/.style={
		rectangle,
		draw,
		minimum width=3.5cm,
		minimum height=1cm
	},
	>=latex,
}

% Also, note that the amsmath package sets \interdisplaylinepenalty to 10000
% thus preventing page breaks from occurring within multiline equations. Use:
\interdisplaylinepenalty=2500
% after loading amsmath to restore such page breaks as IEEEtran.cls normally does.
\usepackage[utf8]{inputenc}
% Useful for displaying quotations
%\usepackage{csquotes}
% Compact lists
%\let\labelindent\relax
%\usepackage{enumitem}
\usepackage{fancyhdr}
\fancyfoot[C]{\thepage}
%tikz figures
%\usepackage{tikz}
%\usetikzlibrary{automata,positioning,chains,shapes,arrows}
%\usepackage{pgfplots}
%\usetikzlibrary{plotmarks}
%\newlength\fheight
%\newlength\fwidth
%\pgfplotsset{compat=newest}
%\pgfplotsset{plot coordinates/math parser=false}

\usepackage{array}
% http://www.ctan.org/tex-archive/macros/latex/required/tools/
%\usepackage{mdwmath}
%\usepackage{mdwtab}
%mdwtab.sty	-- A complete ground-up rewrite of LaTeX's `tabular' and  `array' environments.  Has lots of advantages over
%		   the standard version, and over the version in `array.sty'.
% *** SUBFIGURE PACKAGES ***
%\usepackage[tight,footnotesize]{subfigure}

\usepackage{subfigure}
\usepackage[top=1.5cm, bottom=2cm, right=1.6cm,left=1.6cm]{geometry}
\usepackage{indentfirst}

\usepackage{times}
% make sections titles smaller to save space
%\usepackage{sectsty}
%\sectionfont{\large}
% enable the use of 'compactitem', a smaller 'itemize'
%\usepackage{paralist}

% MP
% to split equations using dmath env
\usepackage{breqn}
% nice rules in tables
\usepackage{booktabs}


%\setlength\parindent{0pt}
\linespread{1}

% MC
\newcommand{\MC}[1]{\textit{\color{red}MC says: #1}}
\newcommand{\AZ}[1]{\textit{\color{blue}AZ says: #1}}
\newcommand{\MP}[1]{\textit{\color{green}MP says: #1}}

\usepackage{placeins}


\setcounter{page}{1}
\usepackage{multirow}
%\mainmatter
%%%%%%%%%%%%%%%%%%%%%%%%%%%%%%%%%%%%%%%%%%
\begin{document}
	%%%%%%%%%%%%%%%%%%%%%%%%%%%%%%%%%%%%%%%%%%
	\title{A deep introspection on Generative Adversarial Networks}
	
	\author{\IEEEauthorblockN{Riccardo Lincetto, Guglielmo Camporese}
		
		\IEEEauthorblockA{Department of Information Engineering, University of Padova -- Via Gradenigo, 6/b, 35131 Padova, Italy \\
			emails: \{riccardo.lincetto, guglielmo.camporese\}@studenti.unipd.it
	}}
	
	\maketitle		
	
	\begin{abstract}
		GANs, namely Generative Adversarial Networks, are a hot topic nowadays. These models have the ability of generating good quality data, learning their distributions on a training set. Image generation in particular benefited from this framework, thanks also to the ease of assessment of the results obtained. Here we review the analysis of GAN models, recalling the necessary results from game theory, and propose a parametrization of the problem, which is then assessed analytically and by means of simulation, on image generation with MNIST dataset. With this parametrization it seems that in some cases it is possible to speed up the training phase.
	\end{abstract}
	
	\section{INTRODUCTION} \label{intro}

Since its rise, deep learning had a great impact on discriminative models. Generative models instead were not affected by this innovation at first, but this trend changed with the introduction of Generative Adversarial Networks (GAN), a powerful framework first introduced in \cite{NIPS2014_5423}. Since then, GAN gained more and more momentum because of the ability of training \textit{deep generative models}, avoiding some of the difficulties encountered in other frameworks \cite{DBLP:journals/corr/Goodfellow17}.

GAN is a sub-class of generative models that use, generally two, \textit{neural networks}: given a training set of sample data, distributed according to a probability density function (pdf) $p_{data}$, the purpose of GAN is to generate samples according to a distribution $p_g$, that mimics $p_{data}$, without explicitly defining it.
As suggested by the name, this is achieved by putting in competition two entities: a generator (G) and a discriminator (D). The task of G is to generate data that can be regarded as true by D, while D has the purpose of correctly distinguishing real from fake data. The classical real-life analogy with this process involves counterfeiters trying to produce fake currency and the police trying to detect it. A graphical representation of the process is depicted in fig.\ref{fig:game}.
This kind of interaction between the two entities can naturally be modeled with a game theoretical approach, where each player has its own strategies and payoffs, but in this paper we will rather talk about costs, as will be discussed in \ref{relatedwork}.
This framework anyway has a major drawback: computing a Nash Equilibrium (NE) requires assuming that the networks have enough capacity, i.e. it has to be done in the non-parametric limit for the networks \cite{NIPS2014_5423}. This implies that the equilibrium point found in practice can be different from the theoretical one, because of intrinsic limits imposed by implementing a network with a finite set of parameters.

In this paper we review some of the literature and explain our need to go back to the origins of GAN, implementing our own version of the code and simulating different scenarios, where discriminator is passed different fake-to-true ratios of images.

The remainder of the paper is organized as follows: a brief overview of the literature is presented in \ref{relatedwork}; a description of our work is then presented in \ref{expsetting}; the obtained results are presented in \ref{results}; finally we discuss our conlcusions in \ref{conclusions}.
	\section{RELATED WORK} \label{relatedwork}

Since their appearance in literature, GANs have been successfully applied to problems of image generation, editing and semi-supervised learning \cite{DBLP:journals/corr/RadfordMC15} \cite{DBLP:journals/corr/ZhangXLZHWM16}.
Results obtained were so promising that the new framework captured the interest of many researchers, leading to a proliferation of various flavors of GAN, each claiming to have better performances on a specific domain.
It's difficult anyway to understand how to compare different GAN models, because of the lack of a consistent metric and the different architectures networks can be designed with, which for each project are related to the corresponding computational budget.
A tentative to define some guidelines to avoid these problems, together with a fair and comprehensive comparison of state-of-the-art GANs, is discussed in \cite{46506}[MISSING ACCENTS ON REF]: what emerges here is that the computational budget plays a major role, allowing bad algorithms to outperform good ones if given enough time; plus, despite the many claims of superiority, there's no empirical evidence that those algorithms are better across all datasets, in fact in most of the cases the original model outperforms the others.
We thus report here a formalization of the original problem \cite{NIPS2014_5423}, modeling it with a game theory approach in a more precise way, as done in \cite{2017arXiv171200679O}.

Deep Convolutional GANs exploit as players (G and D) the learning capability of neural networks. The two anyway have different purposes, so in general they are designed with different architectures (e.g. as in FIG.X).
As suggested in \cite{NIPS2014_5423}, the generator's distribution $p_g$ over data $x$ is learnt defining a prior on input noise variables $p_z(z)$ and representing the mapping to the data space as $G(z;\theta_g)$, where $\theta_g$ stands for the weights of the network.
Similarly for the discriminator, a mapping $D(x,\theta_d)$ can be defined from the data space to a scalar, also referred to as $D(x)$, representing the probability that the input belongs to the true distribution $p_{data}$, i.e. to the training set.
In the formalization of the game then, pure strategies are defined by the sets of possible $\theta_g$ and $\theta_d$, while the utilities functions are the opposite of the loss functions that the networks have to minimize.

For classification tasks, cross-entropy loss function is universally accepted as the best choice, giving good results in terms of learning speed: this is what is usually selected for D in GAN.
The simplest design of GAN uses as loss function for G the opposite of D's, defining thus a zero-sum game (MM-GAN).
Ideally, the number of strategies for each player would be infinite but, as pointed out in  \cite{2017arXiv171200679O}, when using floating point numbers it becomes finite, albeit very large: the game is then finite, implying the existence of a NE, at least in mixed strategies, which however is hard to compute.
It has to be kept into account also the fact that the optimization of a neural network can lead to a Local NE (LNE) because the problem is non-convex.
We can then solve the equilibrium problem by computing the minimax of the payoffs.
Loss functions are defined as:
\begin{align*}
L^{(D)} = -\mathbb{E}_{x \sim p_{data}(x)}[\log D(x)] - \mathbb{E}_{z \sim p_{z}(z)}[\log (1-D(G(z)))]\\
L^{(G)} = - L^{(D)},
\end{align*}
where, the expectations are computed on mini-batches of data.
For the very definition of cross-entropy loss function, $L^{(G)}$ can be simplified because all the samples are generated according to the same distribution: 
\begin{align*}
L^{(G)} = \mathbb{E}_{z \sim p_{z}(z)}[\log (1-D(G(z)))].
\end{align*}
The value of the game is defined as $V(G,D)=-L^{(D)}$.
Fixing a generator $\bar{G}$, the optimal discriminator can be computed maximizing D's utility, which corresponds to $V(\bar{G},D)$:
\begin{align*}
D^*(x) = arg\max\limits_{D} \big\{V(D,\bar{G}) \big\}.
\end{align*}
Assuming that the model has infinite capacity in representing pdfs, we can study the convergence to a NE point in the space of pdf:
\begin{align*}
	V(\bar{G},D) =\\
	 \int_x p_{data(x)} \log(D(x)) dx + \int_z p_{z}(z) \log(1-D(\bar{G}(z))) dz =\\ 
	= \int_x \Big[p_{data(x)} \log(D(x)) + p_{g}(x) \log(1-D(x)) \Big]dx =\\
	= \int_x L\big(x,D(x),\dot{D}(x)\big)dx.
\end{align*}
$D^*(x)$ must satisfy the Euler-Lagrange equation:
\begin{align*}
	\frac{\partial L}{\partial D} = \frac{d L}{d x} \frac{\partial L}{\partial \dot{D}}
\end{align*}
and since $\partial L / \partial \dot{D} = 0$ we get:
\begin{align*}
	\frac{\partial L}{\partial D} = \frac{p_{data}(x)}{D(x)} - \frac{p_g(x)}{1-D(x)} = 0
\end{align*}
that implies:
\begin{align*}
	D^*(x) = \frac{p_{data}(x)}{p_{data}(x) + p_g(x)}.
\end{align*}
Minimizing then over $p_g(x)$, a point of minimum is found for $p_g = p_{data}$, where $V(G^*,D^*)=-log(4)$.

Cross-entropy loss functions are particularly effective for discrimination tasks, but they're not for generation ones because, in the beginning of the training phase, when G is poor and D is able to correctly recognize generated samples, $log(1-D(G(z)))$ \textit{saturates} to zero, thus resulting in a poor gradient. The learning in that case is too slow, but this problem can be avoided changing the loss function for G: instead of minimizing $log(1-D(G(z)))$, we can maximize $log(D(G(z)))$. This is formally defined as a Non-Saturating GAN (NSGAN), where the losses to be minimized are:
\begin{align*}
L^{(D)} = -\mathbb{E}_{x \sim p_{data}(x)}[\log D(x)] - \mathbb{E}_{z \sim p_{z}(z)}[\log (1-D(G(z)))]\\
L^{(G)} = - log(D(G(z))).
\end{align*}

This new game isn't zero-sum any more, but the same equilibrium point of the dynamics can be found as before, thus providing the same theoretical results.

	\section{EXPERIMENTAL SETTING} \label{expsetting}

For our project, we implemented a version of NS-GAN which can be found at [github.com/RicLincio/gt-project]. The training of the two neural networks, depicted in fig.\ref{fig:netG} and fig.\ref{fig:netD}, occurs after mini-batches of data are passed to them: for the discriminator network D, in a mini-batch there are usually the same number of true and of fake images. We instead carried out the training with different configurations, where D is passed different true-to-fake data ratios: this is done in practice defining a parameter $\alpha\in[0,1]$ that represents the portion of true data with respect to the size of the mini-batch.
Considering a zero-sum game, this results in a parametrisation of the loss functions as follows:
\begin{equation*}
	\begin{split}
		L^{(D)} = &-\alpha\cdot\mathbb{E}_{x \sim p_{data}(x)}[\log D(x)]+\\
				  &-(1-\alpha)\cdot\mathbb{E}_{z \sim p_{z}(z)}[\log (1-D(G(z)))];\\
		L^{(G)} = &- L^{(D)}.
	\end{split}
\end{equation*}
Performing the same analysis as before, we found that the optimal discriminator should have the form:
\begin{align*}
	D_{\alpha}^*(x) = \frac{\alpha \cdot p_{data}(x)}{\alpha \cdot p_{data}(x) + (1-\alpha)\cdot p_g(x)}
\end{align*}
and that the loss of the generator is minimized, again, when
\begin{align*}
	p_g = p_{data},
\end{align*}
from which the optimal D can be rewritten as
\begin{align*}
	D_{\alpha}^*(x) = \alpha.
\end{align*}
Supporting these results, it can also be noticed that when setting $\alpha = 0.5$, i.e. when there are the same amount of true and fake images in a mini-batch, results are consistent with what found in \ref{relatedwork}. The value of the game here is:
\begin{align*}
	V(G^*,D^*_\alpha) = -\alpha\cdot\log(\alpha) - (1-\alpha)\cdot\log(1-\alpha).
\end{align*}
As before the equilibrium point can be extended to the case of NS-GAN, though with different losses:
\begin{equation*}
	\begin{split}
		L^{(D)} = &V(G^*,D^*_\alpha) = -\alpha\cdot\log(\alpha) - (1-\alpha)\cdot\log(1-\alpha),\\
		L^{(G)} = &-\log(\alpha).
	\end{split}
\end{equation*}

To evaluate the effects of this parametrisation, we preliminarily trained the NS-GAN models on two different 2D artificial datasets: first on a "sigmoid-shaped" line and then on a circle, as can be seen in \ref{fig:2Dplots}.
The values of $\alpha$ that we tested are $\alpha = \{0.1,0.3,0.5,0.7,0.9\}$.
Once evaluated on the two artificial distribution, we tested our code also on the MNIST dataset on the same values of $\alpha$.
All results are reported in \ref{results}.
	% !TEX root = template.tex

\section{Results}
\label{sec:results}

\MR{In this section, you should provide the numerical results. You are free to decide the structure of this section. As general rules of thumb, use plots to describe your results, showing, e.g., precision, recall and \mbox{F-measure} as a function of the system (learning) parameters. Present the material in a progressive and logical manner, starting with simple things and adding details and explaining more complex behaviors as you go. Also, do not try to explain / show multiple concepts at a time. Try to address one concept at a time, explain it properly, move to the next one.\\
The best results are obtained by generating the graphs in either \texttt{encapsulated postscript (eps)} or \texttt{pdf} formats. To plot your figures, use the \texttt{includegraphics} command.}


	\section{CONCLUSIONS} \label{conclusions}

In this report we presented a review of GAN framework, as presented in the original paper \cite{NIPS2014_5423}, with a game theoretical approach, going through the resolution method to find a Nash equilibrium. We then proposed a parametrization of the loss function by varying the percentage of true images, used for the training in each mini-batch, adapting the analysis previously presented to this new case. Then we tried to apply our own implementation of the model to three different datasets, two of which are generated artificially, while the last one is the MNIST dataset, well known in literature for machine learning and computer vision problems.
Results obtained show that when mini-batches are more balanced, the generator model performs better, but also trainings with less true samples can be performed with good results, which opens the door to the use of GANs even for slightly smaller datasets.

Another possible way to organise mini-batches, that could be investigated in the future, is to introduce randomness in the percentage of true samples per mini-batch: $\alpha$ for example could stand for the probability of passing a true sample.
	
	\bibliography{biblio}
	\bibliographystyle{plain}
	
	
\end{document}