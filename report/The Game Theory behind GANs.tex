\documentclass[10pt, conference, letterpaper]{IEEEtran}

\usepackage{algorithm}
\usepackage{algorithmicx}
\usepackage{algpseudocode}
\usepackage{amsfonts}
\usepackage{amsmath}
\usepackage{amssymb}
\usepackage[ansinew]{inputenc} 
\usepackage{xcolor}
\usepackage{mathtools}
\usepackage{graphicx}
\usepackage{caption}
\usepackage{subcaption}
\usepackage{import}
\usepackage{multirow}
\usepackage{cite}
\usepackage[export]{adjustbox}
\usepackage{breqn}
\usepackage{mathrsfs}
\usepackage{acronym}
\usepackage[keeplastbox]{flushend}
\usepackage{setspace}
\usepackage{stackengine}

\renewcommand{\thetable}{\arabic{table}}
\renewcommand{\thesubtable}{\alph{subtable}}

\DeclareMathOperator*{\argmin}{arg\,min}
\DeclareMathOperator*{\argmax}{arg\,max}

\def\delequal{\mathrel{\ensurestackMath{\stackon[1pt]{=}{\scriptscriptstyle\Delta}}}}

\graphicspath{{./figures/}}
\setlength{\belowcaptionskip}{0mm}
\setlength{\textfloatsep}{8pt}

\newcommand{\eq}[1]{Eq.~\eqref{#1}}
\newcommand{\fig}[1]{Fig.~\ref{#1}}
\newcommand{\tab}[1]{Tab.~\ref{#1}}
\newcommand{\secref}[1]{Section~\ref{#1}}

\newcommand\MR[1]{\textcolor{blue}{#1}}
\newcommand\red[1]{\textcolor{red}{#1}}

%\renewcommand{\baselinestretch}{0.98}
% \renewcommand{\bottomfraction}{0.8}
% \setlength{\abovecaptionskip}{0pt}
\setlength{\columnsep}{0.2in}

% \IEEEoverridecommandlockouts\IEEEpubid{\makebox[\columnwidth]{PUT COPYRIGHT NOTICE HERE \hfill} \hspace{\columnsep}\makebox[\columnwidth]{ }} 

\title{We Rock the Hizzle and Stuff}

\author{Author one$^\dag$, Author two$^\ddag$
\thanks{$^\dag$Author one affiliation, email: \{name.surname\}@dei.unipd.it}
\thanks{$^\ddag$Author two affiliation, email: \{name.surname\}@dei.unipd.it}
\thanks{Special thanks / acknowledgement go here.}
} 

\IEEEoverridecommandlockouts

\begin{document}

\maketitle

\begin{abstract}
Future vehicular communication networks call for new solutions to support their capacity demands, by leveraging the potential of the \mbox{millimeter-wave} (\mbox{mm-wave}) spectrum. Mobility, in particular, poses severe challenges in their design, and as such shall be accounted for. A key question in \mbox{mm-wave} vehicular networks is how to optimize the \mbox{trade-off} between directive Data Transmission (DT) and directional Beam Training (BT), which enables it. In this paper, learning tools are investigated to optimize this \mbox{trade-off}. In the proposed scenario, a Base Station (BS) uses BT to establish a \mbox{mm-wave} directive link towards a Mobile User (MU) moving along a road. To control the BT/DT \mbox{trade-off}, a Partially Observable (PO) Markov Decision Process (MDP) is formulated, where the system state corresponds to the position of the MU within the road link. The goal is to maximize the number of bits delivered by the BS to the MU over the communication session, under a power constraint. The resulting optimal policies reveal that adaptive BT/DT procedures significantly outperform \mbox{common-sense} heuristic schemes, and that specific mobility features, such as user position estimates, can be effectively used to enhance the overall system performance and optimize the available system resources.\\ \MR{This is a sample abstract, just to show as an abstract should be. It is 204 words long, I would say an abstract should not be longer than 250 words. Here, you should briefly state: 1) technical scenario and its importance, 2) what you do in the report / paper and why it is important, 3) if possible, summarize the main results. The abstract should be written in a way that motivates the reader to delve into the paper, but at the same time it should contain enough information to deliver the main message about the paper, so that the reader will now what can be found within the paper even without reading it (as it is the case most of the times). The abstract is a mini-paper on its own and, as such, is a major endeavor to write.}
\end{abstract}

\IEEEkeywords
Mm-Wave, Vehicular Networks, Optimization, Beam Training, Data Transmission, Partially Observable MDP. \MR{A list of keywords defining the tools and the scenario. I would not go beyond {\it six} keywords.}
\endIEEEkeywords


\input{Intro}

% !TEX root = template.tex

\section{Related Work}
\label{sec:related_work}

\MR{The goal of this section is to describe what has been done so far in {\it the} literature. You should focus on and briefly describe the work done in the best papers that you have read. For each you should comment on the paper's contribution, on the good and important findings of such paper and also, 1) on why these findings are not enough and 2) how these findings are improved upon / extended by the work that you do here. At the end of the section, you recap the main paper contributions (one or two, the most important ones) and how these extend / improve upon previous work. If possible, I would make this section no longer than one page, this leads to an overall {\it two pages} including abstract, introduction and related work. I believe this is a fair amount of space in most cases.}\\
\begin{itemize}
\item \MR{\textbf{References:} please follow this {\it religiously}. It will help you a lot. Use {\it bibtex} as the tool to manage the bibliography. A bibtex example file, maned {\tt biblio.bib} is also provided with this package.}

\item \MR{When referring to \textbf{conference / workshop papers}, I recommend to always include the following information: 1) author names, 2) paper title, 3) conference / workshop name, 4) conference / workshop address, 5) month, 6) year. Examples of this are: \cite{Zargham-2011}\cite{Sadler-2006}.}

\item \MR{When referring to \textbf{journal papers}, include the following information: 1) author names, 2) paper title, 3) full journal name, 4) volume, 5) number, 6) month, 7) pages, 8) year. Examples of this are: \cite{Shannon-1948}\cite{Boyd-2011}\cite{Zordan-2014}.}

\item \MR{For \textbf{books}, include the following information: 1) author names, 2) book title, 3) editor and edition, 4) year.}
\end{itemize}
%
\MR{Note that some of the above fields may not be shown when you compile the Latex file, but this depends on the bibliography settings (dictated by the specific Latex style that you load at the beginning of the document). You may decide to include additional pieces of information in a given bibliographic entry, but please, be consistent across all the entries, i.e., use the same fields. Exceptions are in the (rare) cases where some of the fields do not exist (e.g., the paper {\it number} or the {\it pages}).}

% !TEX root = template.tex

\section{Processing Pipeline}
\label{sec:processing_architecture}

\MR{I would start the technical description with a {\it high level} introduction of your processing pipeline. Here you do not have to necessarily go into the technical details of every processing block, this will be done later as the paper develops. What I would like to see here is a description of the general approach, i.e., which processing blocks you used, how these were concatenated, etc. A diagram usually helps.}

\section{Signals and Features}
\label{sec:model}

\MR{Being a machine learning paper, I would put here a section describing the signals you have been working on. If possible, you should describe, in order, 1) the measurement setup, 2) how the signals were \mbox{pre-processed} (to remove noise, artifacts, fill gaps or represent them through a constant sampling rate, etc.). After this, you should describe how {\it feature vectors} were obtained from the \mbox{pre-processed} signals. If signals are {\it time series} this also implies stating the segmentation / windowing strategy that was adopted, to then describe how you obtained a feature vector for each time window. Also, if you also experiment with previous feature extraction approaches, you may want to list them as well, in addition to (and before) your own (possibly new) proposal.}

\section{Learning Framework}
\label{sec:learning_framework}

\MR{Here you finally describe the learning strategy / algorithm that you conceived and used to solve the problem at stake. A good diagram to exemplify how learning is carried out is often very useful. In this section, you should describe the learning model, its parameters, any optimization over a given parameter set, etc. You can organize this section in \mbox{sub-sections}. You are free to choose the most appropriate structure.} 

% !TEX root = template.tex

\section{Results}
\label{sec:results}

\MR{In this section, you should provide the numerical results. You are free to decide the structure of this section. As general rules of thumb, use plots to describe your results, showing, e.g., precision, recall and \mbox{F-measure} as a function of the system (learning) parameters. Present the material in a progressive and logical manner, starting with simple things and adding details and explaining more complex behaviors as you go. Also, do not try to explain / show multiple concepts at a time. Try to address one concept at a time, explain it properly, move to the next one.\\
The best results are obtained by generating the graphs in either \texttt{encapsulated postscript (eps)} or \texttt{pdf} formats. To plot your figures, use the \texttt{includegraphics} command.}



\section{CONCLUSIONS} \label{conclusions}

In this report we presented a review of GAN framework, as presented in the original paper \cite{NIPS2014_5423}, with a game theoretical approach, going through the resolution method to find a Nash equilibrium. We then proposed a parametrization of the loss function by varying the percentage of true images, used for the training in each mini-batch, adapting the analysis previously presented to this new case. Then we tried to apply our own implementation of the model to three different datasets, two of which are generated artificially, while the last one is the MNIST dataset, well known in literature for machine learning and computer vision problems.
Results obtained show that when mini-batches are more balanced, the generator model performs better, but also trainings with less true samples can be performed with good results, which opens the door to the use of GANs even for slightly smaller datasets.

Another possible way to organise mini-batches, that could be investigated in the future, is to introduce randomness in the percentage of true samples per mini-batch: $\alpha$ for example could stand for the probability of passing a true sample.

\bibliography{biblio}
\bibliographystyle{ieeetr}

\end{document}


